\documentclass[11pt,twoside,a4paper]{article}

\def\ThesisYear{2018}

\usepackage{float}
\usepackage{afterpage}
\usepackage[english]{babel}
\usepackage{hyperref}
\usepackage{amsmath}
\usepackage{amssymb}
\usepackage{dsfont}
\usepackage[utf8]{inputenc}
\usepackage{geometry} 
\geometry{a4paper} 
\usepackage{graphicx}
\usepackage{amssymb}
\usepackage{comment}
\usepackage{booktabs} 
\usepackage{array}    
\usepackage{paralist} 
\usepackage{verbatim}
\usepackage{subfig}   
\usepackage{fancyhdr} 
\pagestyle{plain}    
\usepackage{sectsty}
\allsectionsfont{\sffamily\mdseries\upshape} 
\usepackage[nottoc,notlof,notlot]{tocbibind} % Put the bibliography in the ToC
\usepackage[titles,subfigure]{tocloft} % Alter the style of the Table of Contents
\renewcommand{\cftsecfont}{\rmfamily\mdseries\upshape}
\renewcommand{\cftsecpagefont}{\rmfamily\mdseries\upshape} % No bold 
\usepackage{apacite}
\graphicspath{ {./images/} } 
\begin{document}

\section{The B-L-SM model}

It is believed that a $Z^\prime$ boson (Zed prime) might be among the first new discoveries at the TeV scale. A particular array of literature incorporates this Z prime boson in what are called "non-exotic" or "non-anomalous" Minimal $Z^\prime$ model. These models are usually based on an extension of the Standard Model (SM) gauge group with some further U(1) symmetry factor. 

In These models the anomaly cancellation conditions imply the addition of three generations of right-handed neutrinos in the fermion sector, while the subsequent breaking of the extended gauge group provided by an extra singlet Higgs boson, extending the electroweak group's dimensions and generating a massive eigenstate that we'll call the  $Z^\prime$ boson. 

Hence we set out to study the minimal B-L-SM model, a model that not only imcorporates the new Z prime boson but also explains the previously mentioned non vanishing neutrino masses.

As we'll see the addition of the single field will also add a new member in the gauge sector, this gauge field $B^\prime$ will have a specific gauge coupling we'll denote as $g_1^\prime$ but due to mixing/nature of the gauge sector we'll also add a mixing gauge coupling term called $\tilde{g}$. 

Given this the goal of this section will be to explore the fermion, gauge and scalar sectors of the BLSM.   

The model is heavy based on the standard model we just discussed so large portions of it are simply the same.

The B-L-SM model obeys  the gauge symmetry based on the group, $ SU(3)_C  \times SU(2)_L \times U(1)_Y \times U(1)_{B-L}$.  This is model can be decomposed into sectors equivalently to the same sectors of the SM, like in the SM there are 4 sectors. 

These are, the Yang-Mills or gauge sector, the scalar sector, the fermionic sector and the yukawa sector. 

\begin{equation}
\mathcal{L}=\mathcal{L}_{YM}+\mathcal{L}_s+\mathcal{L}_f+\mathcal{L}_Y
\end{equation}

In this model the charges remain the same across all fields, expect with the addition of the charge associated to the $U(1)_{B-L}$ symmetry, $Y_{B-L}$, we'll treat the simplest model where this charge is 0 for the SM higgs field and +2 for the new $\chi$ complex scalar singlet, this is done to retain invariance in the Yukawa terms. All charges can be seen in the following table.  

This stated the covariant derivative would be, 

\begin{equation}
D_\mu = \partial_\mu + igT^\alpha G^\alpha_\mu + igT^\alpha W^a_\mu + ig_1YB_\mu + i(\tilde{g}+g^\prime Y_{B-L})B^\prime_\mu 
\end{equation}


\subsection{The scalar sector}	

Given that we know the quantum numbers of every field and hence the covariant derivative we'll move on to the scalar and gauge sector that as we discussed is expanded from it's SM counterpart with a new scalar singlet field called $\chi$ as given by. The scalar section and gauge section of the Lagrangian stem from, 

\begin{equation}
\mathcal{L}_s = (D^\mu H)^\dagger (D_\mu H) + (D^\mu \chi)^\dagger (D_\mu \chi) - V(H,\chi)
\end{equation}
 
We mentioned that the Higgs mechanism works the same but we'll also the the new field $\chi$ acquires a VEV in the TeV scale. 

However to properly examine the spontaneous symmetry breaking of the model we first must like in the SM ensure that the potential is bound from bellow allowing for continuous a spectrum of energy states. The potential is given by, 

\begin{equation}
 V(H,\chi) = m^2 H^\dagger H +  \mu^2 |\chi|^2 + \begin{pmatrix}
  H^\dagger H && |\chi|^2 \end{pmatrix}  \begin{pmatrix}   
  \lambda_1 && \frac{1}{2}   \lambda_2 \\
  \frac{1}{2}   \lambda_2   && \lambda_3
 \end{pmatrix} \begin{pmatrix}
   H^\dagger H \\ |\chi|^2 
 \end{pmatrix}
\end{equation}

This matrix we see in the potential equation is simply a way to express the self interactions terms in a more visual manner than it's standard quadratic expressions. To ensure the potential is bound from bellow we must ensure that the determinant of this matrix is positive, witch leads to the following restraints on these couplings. 

\begin{equation}
4 \lambda_1 \lambda_2  -  \lambda_3^2 > 0 \quad , \quad \lambda_1 , \lambda_2>0
\end{equation}


We'll return to these conditions later, but having this done we can examine the $\mathcal{L}_s$ components around the vev to observe the symmetry breaking effects, first we must define the vev's that $\chi$ and $H$ will take. 


\begin{equation}
H= 
\begin{pmatrix}
\phi^+ + i  \phi^-  \\
\phi^0_1 +i  \phi_2^0 
\end{pmatrix}
\overset{Electroweak VEV}{\Rightarrow}
\begin{pmatrix}
0  \\
\frac{1}{\sqrt{2}}(v+h(x))
\end{pmatrix}
\end{equation}
and
\begin{equation}
\chi= 
\begin{pmatrix}
\chi
\end{pmatrix}
\overset{"new" VEV}{\Rightarrow}
\begin{pmatrix}
\frac{1}{\sqrt{2}}(x+h^\prime)
\end{pmatrix}
\end{equation}

Where $v$ and $x$ are both positive real numbers. Solving the scalar Lagrangian after the symmetry breaking equates to, 
Expanding these terms we obtain 2 massive eigenstates, one we'll atribute to the recently discovered massive higgs boson but the existence of a second massive state associated to a another quadratic term means that a scalar particle much like the higgs boson is still to be discovered. The massive eigenvalues for the system once we diagonalize the physical base leads us to the following relations for the mass terms squared. 

\begin{equation}
m_{h_1}^2 = \lambda_1 v^2 + \lambda_2 x^2 - \sqrt{(\lambda_1 v^2 - \lambda_2 x^2)^2 + (\lambda_3 x v)^2}
\end{equation}

\begin{equation}
m_{h_2}^2 = \lambda_1 v^2 + \lambda_2 x^2 + \sqrt{(\lambda_1 v^2 - \lambda_2 x^2)^2 + (\lambda_3 x v)^2}
\end{equation}


Solving such equations in order to the vacuum values leads to the set of equations for them

\begin{equation}
v^2 	=\frac{-\lambda_2 m^2 + \frac{\lambda_3}{2} \mu^2}{\lambda_1 \lambda2 - \frac{\lambda_3 }{2}} 
\end{equation}
\begin{equation}
x^2 	=\frac{-\lambda_1 \mu^2 + \frac{\lambda_3}{2} m^2}{\lambda_1 \lambda2 - \frac{\lambda_3 }{2}} 
\end{equation}

Along with these restrains we must also satisfy the conditions stated before to have the potential bound from bellow, combining these leads to

\begin{equation}
\lambda_2 m^2 < \frac{\lambda_3}{2} \mu^2  \quad \text{and} \quad  \lambda_1 \mu^2 < \frac{\lambda_3}{2} m^2
\end{equation}.

This final equation explains something you might have noticed just earlier, namely that $\lambda_1$ and $\lambda_2$ could be both negative and allow for the potential to be bound from bellow, however given the following relations those solutions would lead to both $\mu^2$ and $m^2$ being negative numbers, and therefore unphysical. 


\subsubsection{Gauge eigenstates}

To determine the gauge boson spectrum, we have to expand the scalar kinetic terms again as we did for the SM. We expect that there still exists a massless gauge boson, the photon, whilst the other gauge bosons become massive this is done in the exact same way as before, note that since the extension we are studying is in the Abelian sector of the SM gauge group, so the charged gauge bosons $W^\pm$ will have masses given by their SM expressions, being related to the $SU(2)_L$ Things change however when we examine the terms that would relate to the $Z$ boson,
Firstly let's expand the the Lagrangian Kinetic terms, 

\begin{align}
\begin{split}
(D^\mu H)^\dagger (D_\mu H) + (D^\mu \chi)^\dagger (D_\mu \chi)  = & \frac{1}{2} \partial^\mu h \partial_\mu h 	+ \frac{1}{2} \partial^\mu h^\prime \partial_\mu h^\prime \\ & + \frac{1}{8} (h+v)^2 \left[ g^2 [W_1^\mu - i W_2^\mu ]^2 + \left( gW^\mu_3 - g1B^\mu - \tilde{g} B^{\prime ,\mu}\right)^2   \right] \\  & + \frac{1}{2} (h^\prime +x)^2 (g^\prime_1 2  B^{\prime, \mu})^2
\end{split}
\end{align}

Remember that $Y_{B-L}=-2$ for the $\chi$ field. As we can see the other gauge boson masses are not so simple to identify, because of mixing. we need to find the eigenvalues and vectors of the mixed system, this is easier to shown using mixing angles, one is the old weiberg/weak mixing angle and the other is a new mixing angle that affects only the $Z$ and the $Z^\prime$ boson. 

\begin{equation}
\begin{pmatrix}
B^\mu \\
W_3^\mu \\
B^{\prime, \mu}
\end{pmatrix}=\begin{pmatrix}
\cos(\gamma_\omega) && -\sin(\gamma_\omega) \cos(\gamma^\prime) && \cos(\gamma_\omega) \sin(\gamma^\prime) \\ 
\sin(\gamma_\omega) && \cos(\gamma_\omega) \cos(\gamma^\prime) && -\cos(\gamma_\omega) \sin(\gamma^\prime) \\ 
0 && \sin\gamma^\prime) && \cos(\gamma^\prime)
\end{pmatrix} \begin{pmatrix}
 A^\mu \\ 
 Z^\mu \\
 Z^{\prime,\mu} \\
\end{pmatrix}
\end{equation}

With $\frac{-\pi}{4} < \gamma^\prime < \frac{\pi}{4}$ such that we can define the new mixing angle: 
\begin{equation}
\tan(2\pi) = \frac{2 \tilde{g} \sqrt{g^2 + g^2_1}}{\tilde{g}^2 + 16 (\frac{x}{y})^2 g^{\prime,2}- g^2 - g^2_1}
\end{equation}
The eigenvalues of the system in the physical base would then represent the scare masses leading to the equations 
\begin{equation}
M_A=0 
\end{equation}
\begin{equation}
M_{Z,Z^\prime}=\sqrt{g^2 + g^2_1} \cdot \frac{v}{2} \left[ \frac{1}{2} \left( \frac{\tilde{g}^2 + 16 (\frac{x}{y})^2 g^{' 2}_1}{g^2 + g^2_1} +1  \right) \pm \frac{\tilde{g}}{\sin(2 \gamma^\prime) \sqrt{g^2 + g^2_1}} \right]
\end{equation}
The LEP experiments constrain $| \gamma^\prime |$ to bellow $10^{-3}$. Present constraints on the VEV x allow a generous range of values along the TeV scale. 


\subsection{Fermion sector and mass generation for the neutrinos}
The fermionic Lagrangian in this model is
\begin{align}
\mathcal{L}_f = \sum_{k=1}^{3}  & ( i \overline{q_{kL}} \gamma_\mu D^\mu q_{kL} + \overline{u_{kR}} \gamma_\mu D^\mu u_{kR} 
 +   i \overline{d_{kR}} \gamma_\mu D^\mu d_{kR}  \\  &  + i \overline{l_{kL}} \gamma_\mu D^\mu l_{kL} + i \overline{e_{kR}} \gamma_\mu D^\mu  e_{kR} + i \overline{\nu_{kR}} \gamma_\mu D^\mu \nu_{kr} )
\end{align}
The charges belonging to these sectors are how've seen in the table are the same as in the SM with $\frac{-1}{3}$ in $Y_{B-L}$ for quarks and $-1$ for leptons with no distinction between generations, hence ensuring universality 

This sector interactions should be mostly the same as in the Standard Model however we'll see that in the yukawa section we'll have the addition of several heavy right handed neutrino fields, and the lighter left handed neutrinos that we'll have coupled to the Higgs Field and the new $\chi$ field.  

\begin{align}
\mathcal{L}_Y= & -y^d_{jk}\overline{q}_{jL} d_{kR} H - y_{jk}^u \overline{q}_{jL} u_{kR} \tilde{H} - y^e_{jk} \overline{l}_j e_{kR} H \\ & -  y^\nu_{jk}\overline{l}_{jk} \nu_{kR} \tilde{H} - y^M_{jk}\overline{\nu_R}^c_j v_{kR} \chi + h.c.
\end{align}
Here we have $\tilde{H}=i\sigma^2 H^* $ and $i$ $j$ $k$ are values ranging from 1 to 3.  These are the only allowed gauge invariant terms. In particular, the last term couples the
neutrinos to the new scalar singlet field, $\chi$ , and it allows for the dynamical generation of neutrino masses, as $\chi$ acquires a VEV.

Neutrino mass eigenstates, obtained after applying the see-saw mechanism, with a reasonable choice of Yukawa couplings, the heavy neutrinos can have masses $m_{v_h}  \sim \mathcal{O}(100) GeV$

In the last line of this equations we see the Majorana and dirac masses for the right and left  handed respectively. 

To extract the neutrino masses we have to diagonalise the neutrino mass matrix taken from the couplings in the Lagrangian. 

\begin{equation}
\mathcal{M}=\begin{bmatrix}
0 && m_D \\
m_D && M 
\end{bmatrix}
\end{equation}
where  
\begin{equation}
m_D=\frac{y^\nu}{\sqrt{2}} v  
\end{equation}
\begin{equation}
M=\sqrt{2} y^M x  
\end{equation}
If we ignore inter-generational mixing allowing for each neutrino generation to be diagonalised  independently. Thus, $\nu_L$ and $\nu_R$  can be written as the following linear combination of Majorana mass eigenstates 

\begin{equation}
\begin{pmatrix}
\nu_L \\
\nu_R 
\end{pmatrix}= \begin{pmatrix}
\cos(\alpha_\nu) && -\sin(\alpha_\nu) \\
\sin(\alpha_\nu) && \cos(\alpha_\nu)
\end{pmatrix}= \begin{pmatrix}
\nu_l  \\
\nu_h
\end{pmatrix} 
\end{equation}
Where we have defined the mixing angles to be 
\begin{equation}
\tan(\alpha_\nu)=-\frac{2m_D}{M}
\end{equation}
This makes the real masses for light and heavy neutrinos to be approximately given by,
\begin{equation}
m_{\nu_l}= \approx m_D^2 , 
\end{equation}
\begin{equation}
m_{\nu_h}=\approx M
\end{equation}
\begin{table}[h]
\centering
\begin{tabular}{|l|l|l|l|l|l|l|l|l|}
\hline
 $\phi$& $q_L$  & $u_R$ & $d_R$ & $l_L$  & $e_R$ & $\nu_R$  &  $H$  & $\chi$  \\ \hline
 $SU(3)_c$& 3 & 3 & 3 & 1 & 1 & 1 & 1  & 1  \\
 $SU(2)_L$& 2  & 1 & 1 & 2 & 1 & 1 & 2  & 1 \\
$Y$ & $\frac{1}{6}$ & $\frac{2}{3}$  & $\frac{-1}{3}$  & $\frac{-1}{2}$ & -1 & 0 & $\frac{1}{2}$ & 0 \\
$B-L$ & $\frac{1}{3}$ & $\frac{1}{3}$ & $\frac{1}{3}$  & -1  & -1 &-1  & 0  & 2  \\ \hline
\end{tabular}
\end{table}
\end{document}
